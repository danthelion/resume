%!TEX TS-program = xelatex
%!TEX encoding = UTF-8 Unicode
% Awesome CV LaTeX Template
%
% This template has been downloaded from:
% https://github.com/posquit0/Awesome-CV
%
% Author:
% Claud D. Park <posquit0.bj@gmail.com>
% http://www.posquit0.com
%
% Template license:
% CC BY-SA 4.0 (https://creativecommons.org/licenses/by-sa/4.0/)
%


%%%%%%%%%%%%%%%%%%%%%%%%%%%%%%%%%%%%%%
%     Configuration
%%%%%%%%%%%%%%%%%%%%%%%%%%%%%%%%%%%%%%
%%% Themes: Awesome-CV
\documentclass[]{awesome-cv}
\usepackage{textcomp}
%%% Override a directory location for fonts(default: 'fonts/')
\fontdir[fonts/]

%%% Configure a directory location for sections
\newcommand*{\sectiondir}{resume/}

%%% Override color
% Awesome Colors: awesome-emerald, awesome-skyblue, awesome-red, awesome-pink, awesome-orange
%                 awesome-nephritis, awesome-concrete, awesome-darknight
%% Color for highlight
% Define your custom color if you don't like awesome colors
\colorlet{awesome}{awesome-red}
%\definecolor{awesome}{HTML}{CA63A8}
%% Colors for text
%\definecolor{darktext}{HTML}{414141}
%\definecolor{text}{HTML}{414141}
%\definecolor{graytext}{HTML}{414141}
%\definecolor{lighttext}{HTML}{414141}

%%% Override a separator for social informations in header(default: ' | ')
%\headersocialsep[\quad\textbar\quad]
    \begin{document}
    
%%%%%%%%%%%%%%%%%%%%%%%%%%%%%%%%%%%%%%
%     Profile
%%%%%%%%%%%%%%%%%%%%%%%%%%%%%%%%%%%%%%
\begin{center}
	\headerfirstnamestyle{Daniel} \headerlastnamestyle{Palma} \\
	\vspace{2mm}
	{\faEnvelope\ danivgy@gmail.com} | {\faMapMarker\ Budapest, Hungary} | {\faLink\ www.danielpalma.website}
\end{center}
%%%%%%%%%%%%%%%%%%%%%%%%%%%%%%%%%%%%%%
%     Experience
%%%%%%%%%%%%%%%%%%%%%%%%%%%%%%%%%%%%%%
\cvsection{Experience}
\begin{cventries}
	\cventry
	{Senior Data Engineer \& Scrum Master}
	{Synetiq}
	{Budapest, Hungary}
	{Oct 2016 – Present}
	{\begin{cvitems}
		\item {Built software to extract, clean, aggregate and visualise data from various biometric sensors (EEG, PPG, eye tracking camera).}
		\item {Moved applications to an orchestrated container based infrastructure from bare metal.}
		\item {Planned and executed move from on-premises to cloud based infrastructure.}
		\item {Developed fully automated CI/CD processes based on a cloud native approach.}
		\item {Conducted user research interviews with existing and potential clients.}
		\item {Organised and managed Scrum related meetings and processes.}
		\end{cvitems}}
	\cventry
	{Software Engineer}
	{Telenor}
	{Budapest, Hungary}
	{Apr 2015 – Oct 2016}
	{\begin{cvitems}
		\item {ETL (extract, transform and load) task design and implementation.}
		\item {Time series and geolocation data visualisation}
		\item {Reporting and monitoring tool development}
		\item {Process automatisation}
		\end{cvitems}}
\end{cventries}
\cvsection{Skills}
\begin{cventries}
	\cventry
	{}
	{\def\arraystretch{1.15}{\begin{tabular}{ l l }
		Programming Languages:  & {\skill{ Python, JavaScript, Go, SQL, bash}} \\
		Infrastructure:  & {\skill{ Google Cloud Platform, Linux, Docker, Kubernetes, PostgreSQL}} \\
		Libraries:  & {\skill{ pandas, Luigi, SQLAlchemy, Flask, Vue.js}} \\
		\end{tabular}}}
	{}
	{}
	{}
\end{cventries}

\vspace{-7mm}
\cvsection{Projects}
\begin{cventries}
	\cventry
	{Winning project in the category: Data powered apps. We developed an uplift model based recommender system which was supplemented by a targeting platform that used deep learning to find similar faces in ads to the future client in order to create more engagement.}
	{Telekom Leading Data Hackathon 2017}
	{Spark, Cassandra, OpenCV, Tensorflow, Keras, pandas}
	{bigdata.kibu.hu}
	{}
	
	\vspace{-5mm}
	\cventry
	{Convert text documents to high fidelity audio(books).}
	{doc2audiobook}
	{Python, GCP}
	{github.com/danthelion/doc2audiobook}
	{}
	
	\vspace{-5mm}
	\cventry
	{My personal blog mainly about technology.}
	{Blog}
	{hugo, netlify}
	{www.danielpalma.website}
	{}
	
	\vspace{-5mm}
	\cventry
	{A MacOS menubar application to display currently playing Spotify track information and lyrics.}
	{pyet}
	{Python}
	{github.com/danthelion/pyet}
	{}
	
	\vspace{-5mm}
	\cventry
	{Winner of the Qusp prize. Integration of wearable EEG brain sensors into mobile, PC, and web-based apps and games (including AR/VR), robotic systems, or IoT devices.}
	{IEEE Brain Hackathon 2016}
	{NeuroPype, NeuroScale}
	{abc-accelerator.com/budapest-hackathon}
	{}
	
	\vspace{-5mm}
\end{cventries}

%%%%%%%%%%%%%%%%%%%%%%%%%%%%%%%%%%%%%%
%     Education
%%%%%%%%%%%%%%%%%%%%%%%%%%%%%%%%%%%%%%
\cvsection{Education}
\begin{cventries}
	\cventry
	{BSc in Computer Science}
	{Eötvös Lóránd University}
	{Budapest, Hungary}
	{Sep 2014 – 2018}
	{}
\end{cventries}

\vspace{-2mm}
\ 
\end{document}